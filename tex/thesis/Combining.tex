\chapter{Combining Turing Machines}
\label{chap:combining}

We have defined the notion of multi-tape Turing machines and some basic machines in Chapter \ref{chap:definitions}.
Now we want to construct bigger machines, for example a single-tape machine that moves the head to the right or to a certain symbol.

\section{Control Flow Operators}
\label{sec:control}

We define control flow operators, like ``match'', ``if then else'', ``sequential composition'', and ``while''.
As a result, we get a shallow-embeded language for programming multi-tape Turing machines in an imperative way.


\section{Machine Transformations}
\label{sec:transformations}

There is a problem when we want to combine machines --- for example using sequential composition:
The number of tapes and the alphabets of both machines have to be exactly the same.
How can we re-use a single-tape Turing machine when we want to build a multi-tape machine?
How can we combine two machines with different alphabets?


%%% Local Variables:
%%% TeX-master: "thesis"
%%% End:
