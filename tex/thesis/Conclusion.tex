\chapter{Conclusion}
\label{chap:conclusion}

% What we did and what others did
We have formalised multi-tape Turing machines in Coq.  We developed a framework for programming and proving correctness and time complexity of
multi-tape Turing machines.  We have demonstrated the power of this framework by implementing a multi-tape Turing machine that simulates a two-stack
machine for a variant of the $\lambda$-calculus.  This abstract machine is a variation from the heap machine in Kunze
et~al.~\cite{KunzeEtAl:2018:Formal}.  The two variants of the heap machine differ in that the programs in our version are linearised lists of
commands.  In~\cite{KunzeEtAl:2018:Formal}, the authors show that heap machines can simulate terms of the programming language $L$, which is a subset
of the call-by-value $\lambda$-calculus.  It should be easy to formalise the reduction from the heap machine in~\cite{KunzeEtAl:2018:Formal} to our
version of the heap machine.  By that, we would formalise the reduction from the halting problem of $L$ to the halting problem of multi-tape Turing
machines.  This is, however, outside the scope of this thesis.  It is an ongoing research project~\cite{ForsterLOLA2018}, to implement a Coq library
of undecidability reductions, and this thesis provides one step towards this goal.  The reduction from halting problem of single-tape Turing machines
to the Post correspondence problem (PCP) has already been mechanised in Coq in~\cite{PCPITP}.

\paragraph{Differences to other implementations of Turing machines}
We build on Asperti and Ricciotti`s framework from~\cite{asperti2015} inside the theorem prover Matita, and initially ported their definitions of
tapes and Turing machines to Coq.  We find their inductive definition of the tape in particular practical, because of its symmetric and final nature.
This is in contrast to the implementation of tapes in~\cite{xu2013}, where tapes are split into two halves and the right half contains the current
symbol.  Because of the symmetric nature of tapes, it was extremely easy to define an operator $\MS{Mirror}$ that mirrors the transition function.
The finite nature made it possible to define an always terminating machine that moves the head to the right (or using $\MS{Mirror}$ to the left) end
of the tape.  This is in contrast to~\cite{ciaffaglione2016}, where tapes are implemented as infinite streams of symbols.  Our framework implements
five major improvements on~\cite{asperti2015}.  (1) By introducing labelled machines we make it unnecessary to reason about concrete states of
machines.  The authors in \cite{asperti2015} already note that reasoning about internal states is tedious and therefore do not include the terminating
state in their definition of realisation.  (2) We have introduced a notion of time complexity that relates the inputs to the number of steps needed
for the computation.  (3) By introducing an operator $\Switch$ that generalises sequential composition and conditional, we simplified the verification
of both operators and also introduced a useful operator that was used throughout the thesis.  (4) We implemented general lifting operators that make
it possible to compose small machines (w.r.t.\ the alphabet and number of tapes) to fairly complex machines.  At this point, composing compound
machines is reasonably easy, but we (5) have introduced another layer of abstraction.  We have made it possible to directly manipulate values of
encodable types.  The on-paper design and implementation in Coq of the heap machine simulator was straight-forward.


% Norrish`s comment about the "daunting prospect".
In~\cite{norrish2011mechanised}, Norrish concludes that:
\begin{quote}
  If register machines are unappealing because of their general fiddliness, Turing machines are an even more daunting prospect.
\end{quote}
Certainly, Turing machines are not appealing.  We have spent considerable efforts (ca.\ one year) to make programming and verifying Turing machines
feasible.  Interestingly, we ended up at a point, where programming and verifying Turing machines can be (in some sense) \textit{easier} than for
register machines, because register machines are restricted to natural numbers.


\paragraph{Duality correctness and time complexity}
We noted that our notions of realisation and termination/running time are dual in a sense.  The weak notion of realisation says that \textit{if} the
machine terminates, then the output is correct w.r.t.\ a correctness relation $R$.  On the other side, a machine terminates in a termination relation
$T$ if all pairs of input tapes $t$ and step numbers $k$, the machine terminates in $k$ with the input $t$.  Realisation is monotone
(cf.~Lemma~\ref{lem:Realise_monotone}) and termination is anti-monotone (cf.~Lemma~\ref{lem:TerminatesIn_monotone}).  We find it remarkable that we
use an inductive correctness relation for $\MS{While}$ (cf.~Lemma~\ref{lem:While_Realise}) and a co-inductive termination relation
(cf.~Lemma~\ref{lem:While_TerminatesIn}).  We use induction to show correctness and co-induction for termination of instances of $\While$.


\paragraph{Similarity of realisation and weak Hoare triples}
As already noted in~\cite{ciaffaglione2016}, the notion of realisation is similar to the classic Hoare logic widely used in program verification.  For
example, consider the classic Hoare proof rule for sequential composition, and had the corresponding relational rule (for unlabelled machines
$M_1,M_2 : \TM_\Sigma^n$):
\[
  \inferrule{\{A\}~P_1~\{B\} \and \{B\}~P_2~\{C\}}{\{A\}~P_1 \Seq P_2~\{C\}}
  \qquad
  \inferrule{M_1 \Realise R_1 \and M_2 \Realise R_2}{M_1 \Seq M_2 \Realise R_1 \circ R_2}
\]
We encode preconditions and postconditions inside correctness relations.  This means that if the precondition is not true for the input tape $t$, then
$R~t~(l,t')$ is logically equivalent to $\True$.  Sequential composition of machines amounts to relational composition of correctness relations
(cf.~Lemma~\ref{lem:Seq_RealiseIn}).  However, we are not aware of a Hoare-style logical calculus for reasoning about termination in a concrete number
of steps that is dual to classical Hoare logic, like in our duality between realisation and termination.

\paragraph{Problems of the framework}
The biggest problem of this framework is that encodability of types can be ambiguous.  For example, there are more than three ways how to encode
natural numbers on the alphabet of the heap machine simulator (cf.~Section~\ref{sec:Lookup}).  We had to mentally keep track of in which encoding a
value is encoded on a tape, and to translate values from one encoding to another.


\paragraph{Comparison of proof assistants}
Asperti and Ricciotti\cite{asperti2015} propose the formalisation of Turing machines as a benchmark for comparing proof assistants.  We think that the
formalisation and usability of finite sets could be a benchmark for itself.  However, the task ``formalise Turing machines in proof assistant $X$'' is
rather broad.  There are many different mathematical formalisations of Turing machines, and some might be easier to implement in one or the other
proof assistant.  For example, Isabelle does not support dependent types, so this concrete formalisation of Turing machines in this thesis would not
be possible in Isabelle.  Dependent types are quite central in our formalisation of Turing machines.  For example, defining the $\Switch$ without them
would probably be considerable harder.



\paragraph{Future work}
When we defined the notion of value-containment (cf.~Section~\ref{sec:value-containment}), we had future work in mind where we formalise space-usage
of machines.  We were careful to avoid memory-leaks in the machines, but have not yet formalised this aspect of correctness.  We can strengthen the
correctness relations with commitments about the space-usage of each tape.  Asperti and Ricciotti`s inductive definition of tapes is very helpful in
this regard, because their tapes never decrease the number of symbols.  This means that the total space usage of a tape is just the number of symbols
on the tape.  Our idea is that we parametrise the notion of value-containment over the length $l$ of the ``rest list'' on the left, and write
$t \simeq_{l} x$.  Note that, by definition, there are no symbols beyond the stop symbol on the right side of the tape, so the total size of the tape
only depends of the length of the encoding and the length of the left rest.  For example, $\MS{CaseNat}$ does not change the amount of totally
allocate memory, but decreases the length of the encoding and increases the length of the rest by one.  On the other side, $\MS{ConstrS}$ ``consumes''
on rest symbol, i.e.\ decreases the length of the rest by one, and increases the length of the encoding by one.  Thus, if the rest is empty,
$\MS{ConstrS}$ allocates one new symbol.

Further future work could be to show that the running time function of the simulator is polynomial w.r.t.\ the number of steps and the length of the
encoding of the initial heap machine state, see~\cite{ForsterLOLA2017}.  We could also formalise the reduction from multi-tape Turing machines to
single-tape Turing machines, and from single-tape Turing machines with arbitrary finite alphabet to single-tape Turing machines with a binary
alphabet.  The framework can be used to program other simulator machines, for example, for the ``naive'' substitution-based machine
in~\cite{KunzeEtAl:2018:Formal}.  We could implement a universal Turing machines as in~\cite{asperti2015}, and formalise results of computationally
and complexity theory, for example the undecidability of the halting problem and Rice`s theorem.  The opposite reduction from multi-tape Turing
machines to $L$, i.e.\ programming an $L$ expression that simulates multi-tape Turing machines, is also open for future work.  This should be a ``less
daunting prospect'', because there is a framework for verified extraction of Coq terms to expressions of $L$, see~\cite{LExtractITP}.


%%% Local Variables:
%%% TeX-master: "thesis"
%%% End:
